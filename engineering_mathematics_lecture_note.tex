\documentclass[unfonts,oneside,a4paper]{oblivoir}

\usepackage{kotex}
\usepackage{microtype}
\usepackage{bm}

\usepackage{mathpazo}
\setmainfont{TeX Gyre Pagella}
\setmainhangulfont[ItalicFont={*},ItalicFeatures={FakeSlant=.167}]{NanumMyeongjo}

\usepackage{amsmath,mathtools,amssymb,amsthm,thmtools}

\theoremstyle{definition}
\newtheorem{definition}{Definition}

\theoremstyle{theorem}
\newtheorem{theorem}{Theorem}

\theoremstyle{remark}
\newtheorem*{myremark}{My remark}

\theoremstyle{remark}
\newtheorem*{example}{Example}

\theoremstyle{remark}
\newtheorem*{homework}{Homework}

\declaretheoremstyle[
spaceabove=6pt, spacebelow=6pt,
headfont=\normalfont\itshape,
notefont=\mdseries, notebraces={(}{)},
bodyfont=\normalfont,
postheadspace=1em,
headpunct={.},
qed=$\blacksquare$,
numbered=no
]{solstyle}
\declaretheorem[style=solstyle]{solution}

\usepackage{diffcoeff}
\diffset[roman = true]

\renewcommand{\vec}[1]{\bm{\mathit{#1}}}
\newcommand{\vecz}{\bm{\mathrm{0}}}
\newcommand{\dd}{\mathrm{d}}
\DeclareMathOperator{\Span}{Span}
\DeclareMathOperator{\Null}{N}
\DeclareMathOperator{\Image}{Im}
\DeclareMathOperator{\Range}{R}

\title{공학 수학\\강의 노트}

\author{이재호\\\href{mailto:jaeho.lee@snu.ac.kr}{\texttt{jaeho.lee@snu.ac.kr}}}

\date{마지막 수정: \today}

\begin{document}

\maketitle

\setcounter{section}{6}
\reversemarginpar{}
\section{Linear Algebra: Matrices, Vectors, Determinants. Linear Systems}

Field는 수학과 물리학에서 지칭하는 대상이 다르다:\marginpar{\small2018년 9월 3일}
\begin{itemize}
    \item \textbf{체體}: 실수체, 복소수체
    \item \textbf{장場}: 전자기장, 벡터장
\end{itemize}

\begin{definition}
    체는 $+, -, \times, \div$에 대해 닫혀 있는 수 집합을 말한다.
\end{definition}

\begin{example}
    \leavevmode
    \begin{itemize}
        \item $\mathbb Q$는 조밀(dense)하다.
        \item $\mathbb R$은 꽉차(complete)있다.
        \item $\mathbb C$는 대수적으로 닫혀(closed)있다.
        \item $\mathbb Q \bigl(\sqrt 2\bigr) = \bigl\{a + b \sqrt 2\,\bigm|\,a, b \in \mathbb Q\bigr\}$
        \item $\mathbb Z_p = \{0, 1, \dots, p - 1\}$
    \end{itemize}
\end{example}

\begin{definition}
    벡터 공간은 (roughly) 덧셈($+$)과 (어떤 체에 속하는) 상수배가 정의된 집합이다.
\end{definition}

\begin{example}
    \leavevmode
    \begin{itemize}
        \item $\mathbb R^2 = \left\{\begin{pmatrix} a \\ b \end{pmatrix}\,\middle|\,a, b \in \mathbb R\right\}$은 덧셈에 대해 닫혀 있고, $k \in \mathbb R$의 곱에 대해 닫혀 있으므로 실수체에 대한 벡터 공간이다.
        \item $\mathbb R^n$과 $\mathbb C^n$은 실수체에 대한 벡터 공간이면서 유리수체에 대한 벡터 공간이다.
            이를 각각 $\mathbb Q$--벡터 공간, $\mathbb R$--벡터 공간이라고 한다.
        \item $\mathcal C^0_I = \{f: I \rightarrow \mathbb R \mid f \text{는 연속 함수},\ I \subset \mathbb R\}$에서는, $x \mapsto \sin x$가 하나의 벡터이다.
        \item $\mathcal C^n_I = \bigl\{f: I \rightarrow \mathbb R\,\bigm|\,\exists f^{(n)},\ f^{(n)} \text{는 연속 함수},\ I \subset \mathbb R\bigr\}$이며, $\mathcal C^0_I > \mathcal C^1_I > \dots > \mathcal C^\infty_I$이다.
    \end{itemize}
\end{example}

\setcounter{definition}{1}
\begin{definition}
    체 $F$에 대한 벡터 공간 $V$의 원소 $\vec v_1, \dots, \vec v_n \in V$에 대해서,
    \begin{enumerate}
        \item $c_1, \dots, c_k \in F$일 때
            \[
                c_1 \vec v_1 + \dots + c_k \vec v_k
            \]
            는 $\vec v_1, \dots, \vec v_k$의 일차 결합, 혹은 선형 결합(linear combination)이라고 한다.
        \item $\vec v_1, \dots, \vec v_k$의 선형 생성
            \[
                \Span \{\vec v_1, \dots, \vec v_k\} = \left< \vec v_1, \dots, \vec v_k \right>
            \]
            은 모든 일차 결합의 집합이다.
            예를 들어, 어떤 평면 상의 평행하지 않은 두 벡터는 그 평면을 선형 생성한다.
        \item $W \subset V$이면서 $W$가 벡터 공간이면, $W$는 $V$의 부분 공간이라고 하며 $W < V$로 표기한다.
            예를 들어, 실수 평면으로 나타내어지는 $\mathbb R^2$ 벡터 공간의 부분 집합인 원점을 지나는 직선은 벡터 공간이므로 부분 공간이다.
            반면, 원점을 지나지 않는 직선은 부분 집합이지만 벡터 공간은 아니므로 부분 공간이 아니다.
        \item $W = \Span \{\vec v_1, \dots, \vec v_k\} < V$일 때, $\vec v_1, \dots, \vec v_k$는 $W$의 생성자(generator)라고 한다.
        \item $\vec v_1, \dots, \vec v_k$가 일차 종속(linearly dependent)라는 것은 어느 하나가 다른 벡터들의 일차 결합이라는 것이다.
            일차 종속이 아니면 일차 독립(linearly independent)이라고 한다.
            예를 들어, $\mathbb R^3$에서 $(1, 2, 3), (4, 5, 6), (7, 8, 9)$는 일차 종속인 반면, $(1, 2, 3), (4, 5, 6)$은 일차 독립이다.
    \end{enumerate}
\end{definition}

\begin{theorem}
    체\marginpar{\small 2018년 9월 5일} $F$에 대한 벡터 공간 $V$의 원소들 $\vec v_1, \dots, \vec v_k$가 일차 독립이라는 것은, $c_1, \dots, c_k \in F$일 때
    \[
        c_1 \vec v_1 + \dots + c_k \vec v_k = \vecz\ \Rightarrow\ c_1 = \dots = c_k = 0
    \]
    이라는 것과 동치이다.
\end{theorem}

\begin{proof}
    $(\Rightarrow)$ (재배열 가능하여) $c_k \neq 0$이라고 가정하자.
    그렇다면,
    \[
        \vec v_k = - \left(\frac{c_1}{c_k} \vec v_1 + \dots + \frac{c_{k - 1}}{c_k} \vec v_{k - 1}\right)
    \]
    이므로 $\vec v_1, \dots, \vec v_k$가 일차 독립이라는 가정에 모순된다.

    $(\Leftarrow)$ $\vec v_1, \dots, \vec v_k$가 일차 종속이라고 가정하자.
    즉, (재배열 가능하여) 어떤 $a_1, \dots, a_{k - 1} \in F$가 존재해서,
    \[
        \vec v_k = a_1 \vec v_1 + \dots + a_{k - 1} \vec v_{k - 1}
    \]
    이다.
    그렇다면
    \[
        a_1 \vec v_1 + \dots + a_{k - 1} \vec v_{k - 1} + (-1) \vec v_k = \vecz
    \]
    이므로 모순이다.
\end{proof}

\begin{example}
    \leavevmode
    \begin{itemize}
        \item $a, b, c \in \mathbb R$에 대해
            \[
                a(1, 2, 3) + b(4, 5, 6) + c(7, 8, 10) = (0, 0, 0)
            \]
            을 만족하는 $a, b, c$는 0밖에 없으므로 $(1, 2, 3), (4, 5, 6), (7, 8, 10)$은 (일차) 독립이다.
        \item $\vecz, \vec v_1, \dots, \vec v_k$는 (일차) 종속이다.
        \item $\vec v_1, \vec v_2, \vec v_3, \vec v_4$가 독립이면 $\vec v_1, \vec v_2, \vec v_3$ 또한 독립이다.
            (왜냐하면 $\vec v_1, \vec v_2, \vec v_3$ 가 종속이면 $\vec v_3 = c_1 \vec v_1 + c_2 \vec v_2 + 0 \vec v_4$이기 때문이다.)
    \end{itemize}
\end{example}

\begin{definition}
    벡터 공간 $V$의 부분 공간 $W$가 일차 독립인 $\vec v_1, \dots, \vec v_k$에 의해 생성될 때, $\{ \vec v_1, \dots, \vec v_k\}$를 $W$의 기저(basis)라고 한다.
\end{definition}

\begin{theorem}
    $W$가 유한 생성 (finitely generated) (부분) 공간이면 $W$의 기저의 원소의 개수는 동일하다.
    이 때, 기저의 원소의 개수를 $W$의 차원(dimension)이라고 하며, $\dim W$로 표기한다.
\end{theorem}

\begin{example}
    \leavevmode
    \begin{itemize}
        \item $\mathcal P_n = \left\{n\text{차 이하 실계수 다항식}\right\}$일 때, $f \in \mathcal P_n$는 항상 $\mathcal B = \{1, x, \dots, x^n\}$의 원소의 상수배의 합으로 나타낼 수 있다.
            또한,
            \[
                c_0 \cdot 1 + c_1 x + \dots + c_n x^n = 0
            \]
            이면 $c_0 = \dots c_n = 0$이므로, $\mathcal B$는 기저이다.
        \item $\mathcal P = \left\{\text{모든 다항식}\right\}$의 기저는 $\left\{1, x, x^2, \dots\right\}$이며 $\dim \mathcal{P} = \infty$이다.
    \end{itemize}
\end{example}

\begin{homework}
    \leavevmode
    \begin{enumerate}
        \item $a_1, a_2, a_3$는 서로 다른 실수이다.
            이 때, $\left(1, a_1, a_1^2\right), \left(1, a_2, a_2^2\right), \left(1, a_3, a_3^2\right)$이 독립임을 보여라.
        \item $a_1, \dots, a_n$는 서로 다른 실수이다.
            이 때, $\left(1, a_1, a_1^2\right), \dots, \left(1, a_n, a_n^2\right)$이 독립임을 보여라.
    \end{enumerate}
\end{homework}

\begin{theorem}\label{thm3}
    체 F에 대한 벡터 공간 $V$의 독립인 $\vec v_1, \dots, \vec v_k$에 대해서, $\vec w \in V \setminus \left<\vec v_1 , \dots, \vec v_k\right>$이면 $\vec v_1, \dots, \vec v_k, \vec w$는 독립이다.
\end{theorem}

\begin{proof}
    $c_1, \dots, c_k, a \in F$일 때,
    \[
        c_1 \vec v_1 + \dots + c_k \vec v_k + a \vec w = \vecz
    \]
    이라고 가정하자.
    만약 $a \neq 0$이라면 
    \[
        \vec w = -\left(\frac{c_1}{a} \vec v_1 + \dots + \frac{c_k}{a} \vec v_k\right) \in \left<\vec v_1, \dots, \vec v_k\right>
    \]
    이므로 모순이다.
    따라서 $a = 0$이며, $\vec v_1, \dots, \vec v_k$는 독립이므로 $c_1 = \dots = c_k = a = 0$이다.
    그러므로 $\vec v_1, \dots, \vec v_k, \vec w$는 독립이다.
\end{proof}

Theorem~\ref{thm3}에 따르면, 독립인 벡터들의 선형 생성에 포함되지 않는 벡터 또한 이들과 독립이다.
이에 따라 어떤 유한 생성 벡터 공간 $V$의 독립인 벡터들 $\vec v_1, \dots, \vec v_k$가 주어졌을 때, 선형 생성에 속하지 않는 벡터 $\vec w_{k + 1}, \dots, \vec w_{\dim V}$를 순차적으로 추가해서 기저를 구성할 수 있다.
이를 기저 확장(basis extension)이라고 부른다.

\begin{theorem}
    체 $F$에 대한 벡터 공간 $V$의 부분 공간 $W_1, W_2$가 주어졌을 때, $W_1 \cap W_2$ 또한 $V$의 부분 공간이다.
\end{theorem}

\begin{proof}
    $\vec v_1, \vec v_2 \in W_1$이면서 $\vec v_1, \vec v_2 \in W_2$이면, 각각 벡터 공간이므로 임의의 $c_1, c_2 \in F$에 대해서 $c_1 \vec v_1 + c_2 \vec v_2 \in W_1$이면서 $c_1 \vec v_1 + c_2 \vec v_2 \in W_2$이다.
    즉,
    \[
        \vec v_1, \vec v_2 \in W_1 \cap W_2\ \Rightarrow\ \forall c_1, c_2 \in F \quad c_1 \vec v_1 + c_2 \vec v_2 \in W_1 \cap W_2
    \]
    이다.
    따라서, $W_1 \cap W_2 < V$가 성립한다.
\end{proof}

\begin{definition}
    벡터 공간 $V$의 부분 집합 $W_1, W_2$에 대해서 $W_1 + W_2$를 다음과 같이 정의한다:
    \[
        W_1 + W_2 = \{\vec v_1 + \vec v_2 \mid \vec v_1 \in W_1, \vec v_2 \in W_2\}.
    \]
\end{definition}

\begin{example}
    좌표 평면으로 나타내어지는 벡터 공간 $\mathbb R^2$의 부분 집합(이면서 부분 공간인)인 원점을 지나는 직선을 $W_1$이라고 하자.
    원점이 아닌 한 점만을 원소로 하는 집합 $W_2 \subset \mathbb R^2$가 주어졌을 때, $W_1 + W_2$는 $W_1$의 직선을 $W_2$의 (유일한) 원소의 점으로 평행 이동한 직선을 나타낸다.
\end{example}

\begin{theorem}
    체 $F$에 대한 벡터 공간 $V$의 부분 공간 $W_1$과 $W_2$에 대해서, $W_1 + W_2$도 $V$의 부분 공간이다.
\end{theorem}

\begin{proof}
    임의의 $\vec v_1 \in W_1 + W_2$에 대해 어떤 $\vec w_1 \in W_1$과 $\vec w_2 \in W_2$가 존재하며, 마찬가지로 임의의 $\vec v_2 \in W_1 + W_2$에 대해 어떤 $\vec u_1 \in W_1$과 $\vec u_2 \in W_2$가 존재한다.
    $W_1$과 $W_2$는 부분 공간이므로 임의의 $c_1, c_2 \in F$에 대해
    \begin{align*}
        c_1 \vec w_1 + c_2 \vec w_2 &\in W_1\\
        c_1 \vec u_1 + c_2 \vec u_2 &\in W_2
    \end{align*}
    이고,
    \begin{equation*}
        c_1 \vec v_1 + c_2 \vec v_2 = (c_1 \vec w_1 + c_2 \vec w_2) + (c_1 \vec u_1 + c_2 \vec u_2)
    \end{equation*}
    이므로,
    \begin{equation*}
        c_1 \vec v_1 + c_2 \vec v_2 \in W_1 + W_2
    \end{equation*}
    이다.
    따라서 $W_1 + W_2 < V$이다.
\end{proof}

\begin{theorem}
    체\marginpar{\small 2018년 9월 10일} $F$에 대한 유한 생성 벡터 공간 $U$와 $W$에 대해, 다음이 성립한다:
    \begin{equation*}
        \dim (U + W) = \dim U + \dim W - \dim (U \cap W).
    \end{equation*}
\end{theorem}

\begin{proof}
    $l = \dim (U \cap W)$, $n = \dim U - l$, $m = \dim W - l$이라 하고, $U \cap W$의 기저를 $\mathcal B_\cap = \{\vec v_1, \dots, \vec v_l\}$라고 하자.
    $\mathcal B_\cap$을 확장해 $U$의 기저 $\mathcal B_U = \{\vec v_1, \dots, \vec v_l, \vec u_1, \dots, \vec u_n\}$와 $W$의 기저 $\mathcal B_W = \{\vec v_1, \dots, \vec v_l, \vec w_1, \dots, \vec w_m\}$을 구성할 수 있다.

    이제 $\mathcal B_U \cup \mathcal B_W = \{\vec v_1, \dots, \vec v_l, \vec u_1, \dots, \vec u_n, \vec w_1, \dots, \vec w_m\}$\footnote{이를 위해서는 $\vec u_1, \dots, \vec u_n$과 $\vec w_1, \dots, \vec w_m$에 겹치는 원소가 없어야 한다.
    이는 $\mathcal B_\cap = \mathcal B_U \cap \mathcal B_W$와 동치인데, 자명할수도 있겠지만 증명 후에 다루었다.}가 $U + W$의 기저임을 보이자.
    $U + W$의 원소 $\vec v$를 고르면, $\vec v = \vec u + \vec w$가 되는
    \[
        \vec u = a_1 \vec v_1 + \dots + a_l \vec v_l + b_1 \vec u_1 + \dots + b_n \vec u_n
    \]
    와
    \[
        \vec w = a'_1 \vec v_1 + \dots + a'_l \vec v_l + c_1 \vec w_1 + \dots + c_m \vec w_m
    \]
    가 존재하고, 이러한 $a_1, \dots, a_l, b_1, \dots, b_n, a'_1, \dots, a'_l, c_1, \dots, c_m \in F$는 유일하다.
    따라서,
    \begin{align*}
        \vec v &= \vec u + \vec w\\
               &= \left(\sum_{i = 1}^l a_i \vec v_i + \sum_{j = 1}^n b_j \vec u_j\right) + \left(\sum_{i = 1}^l a'_i \vec v_i + \sum_{k = 1}^m c_k \vec w_k\right)\\
               &= \sum_{i = 1}^l \left(a_i + a'_i\right) \vec v_i + \sum_{j = 1}^n b_j \vec u_j + \sum_{k = 1}^m c_k \vec w_k
    \end{align*}
    이므로 $\vec v$는 $\Span (\mathcal B_U \cup \mathcal B_W)$의 원소이고, $\mathcal B_U \cup \mathcal B_W$는 $U + W$를 생성한다.

    이제
    \begin{equation*}
        \sum_{i = 1}^l a_i \vec v_i + \sum_{j = 1}^n b_j \vec u_j + \sum_{k = 1}^m c_k \vec w_k = \vecz
    \end{equation*}
    이라고 하자.
    $\vec u_j$항을 제외하고 모두 우변으로 이항하면
    \begin{equation*}
        \sum_{j = 1}^n b_j \vec u_j = \sum_{i = 1}^l -a_i \vec v_i + \sum_{k = 1}^m -c_k \vec w_k
    \end{equation*}
    이 되고, $\vec z = \sum_{j = 1}^n b_j \vec u_j$라고 두자.
    \begin{align*}
        \sum_{j = 1}^n b_j \vec u_j \in \Span \mathcal B_U = U\\
        \sum_{i = 1}^l -a_i \vec v_i + \sum_{k = 1}^m -c_k \vec w_k \in \Span \mathcal B_W = W
    \end{align*}
    이므로, $\vec z \in U \cap W = \Span \mathcal B_\cap = \left<\vec v_1, \dots, \vec v_l\right>$이다.
    즉
    \[
        \vec z = d_1 \vec v_1 + \dots + d_l \vec v_l
    \]
    이 되는 $d_1, \dots, \vec d_l \in F$가 유일하게 존재한다.
    그런데 $\mathcal B_U$와 $\mathcal B_W$는 일차 독립이므로 $j \in \{1, \dots, l\}$에 대해 $b_j = 0$이고 $k \in \{1, \dots, m\}$에 대해 $c_k = 0$이다.
    이에 따라 $\vec z = \vecz$이 되어 $i \in \{1, \dots, l\}$에 대해서도 $a_i = 0$이다.
    그러므로 $\mathcal B_U \cup \mathcal B_W$은 일차 독립이다.

    $\mathcal B_U \cup \mathcal B_W$는 $U + W$를 생성하고 일차 독립이므로, $U + W$의 기저이다.
    결국
    \begin{align*}
        \dim (U + W) &= |\mathcal B_U \cup \mathcal B_W|\\
                     &= l + n + m\\
                     &= (l + n) + (l + m) - l\\
                     &= \dim U + \dim W - \dim (U \cap W)\qedhere
    \end{align*}
\end{proof}

\begin{myremark}
    위 증명에서 $\mathcal B_\cap = \mathcal B_U \cap \mathcal B_W$임을 보이자.\footnote{참고로, $\mathcal B_U \setminus \mathcal B_W$는 $U \setminus W$의 기저가 아니다.
    $\vecz \in U \cap W$임에 따라 $\vecz \notin U \setminus W$이므로, $U \setminus W$는 벡터 공간이 아니기 때문이다.}
    위 증명이 성립하기 위해서는 $\{\vec u_1, \dots, \vec u_n\}$와 $\{\vec w_1, \dots, \vec w_m\}$의 모든 원소가 달라야하기 때문이다.
    먼저 $\mathcal B_\cap \subseteq \mathcal B_U \cap \mathcal B_W$임은 자명하다.
    $\mathcal B_U \cap \mathcal B_W$에 속하는 원소 $\vec x$를 고르자.
    $\mathcal B_U \cap \mathcal B_W \subseteq U \cap W$이므로, $\vec x \in \Span \mathcal B_\cap$이다.
    나아가 $\mathcal B_U$와 $\mathcal B_W$의 원소인 $\vec x$는 $\mathcal B_\cap$의 모든 원소들과 독립이므로, $\vec x \in \mathcal B_\cap$이어야 한다.
    따라서 $\mathcal B_U \cap \mathcal B_W \subseteq B_\cap$이다.
\end{myremark}

\begin{definition}
    체 $F$에 대한 벡터 공간 $V, W$에서, 임의의 $\vec v_1, \vec v_2 \in V$와 $c_1, c_2 \in F$에 대해
    \begin{equation*}
        L(c_1 \vec v_1 + c_2 \vec v_2) = c_1 L(\vec v_1) + c_2 L(\vec v_2)
    \end{equation*}
    을 만족하는 함수 $L: V \rightarrow W$을 선형 변환(linear transformation) 혹은 선형 사상(linear map)이라고 한다.
\end{definition}

\begin{example}
    \leavevmode
    \begin{enumerate}
        \item $L: \mathbb R^2 \rightarrow \mathbb R^3$에 대해,
            \begin{equation*}
                L \begin{pmatrix}
                    x \\ y
                    \end{pmatrix} = \begin{pmatrix}
                    x + 2y \\ 3x + 4y \\ 5x + 6y
                    \end{pmatrix} = \begin{pmatrix}
                    6 & 2 \\ 3 & 4 \\ 5 & 6
                    \end{pmatrix} \begin{pmatrix}
                    x \\ y
                \end{pmatrix}
            \end{equation*}
            은 선형 변환이다.
        \item $L: \mathbb R^2 \rightarrow \mathbb R^3$에 대해,
            \begin{equation*}
                L \begin{pmatrix}
                    x \\ y
                    \end{pmatrix} = \begin{pmatrix}
                    x + 2y \\ 3x + 4y \\ 5x + 6y + 1
                    \end{pmatrix} = \begin{pmatrix}
                    6 & 2 \\ 3 & 4 \\ 5 & 6
                    \end{pmatrix} \begin{pmatrix}
                    x \\ y
                    \end{pmatrix} + \begin{pmatrix}
                    0 \\ 0 \\ 1
                \end{pmatrix}
            \end{equation*}
            은 선형 변환이 아니다.
        \item 어떤 사상 $L: \mathbb R^n \rightarrow \mathbb R^m$이 존재한다는 것은 어떤 $m \times n$ 행렬 $A$가 존재해서 $L(\vec x) = A \vec x$라는 것이다.
        \item $f \mapsto f'$의 미분 연산자 $L: \mathcal C^\infty_I \rightarrow \mathcal C^\infty_I$는 $L(c_1 f_1 + c_2 f_2) = c_1 f'_1 + c_2 f'_2 = c_1 L(f_1) + c_2 L(f_2)$이므로 선형 변환이다.
        \item $f(x) \mapsto \int_0^x f(t)\, \dd t$의 적분 연산자 $L: \mathcal C^\infty_I \rightarrow \mathcal C^\infty_I$ 또한 선형 변환이다.
    \end{enumerate}
\end{example}

\begin{definition}
    두 벡터 공간 $V$와 $W$ 간의 선형 변환 $L: V \rightarrow W$에 대해 다음이 정의된다:
    \begin{enumerate}
        \item $L$의 핵(kernel) 혹은 영 공간(null space)
            \begin{equation*}
                \ker L = \Null (L) = \{\vec v \in V \mid L(\vec v) = \vecz\}.
            \end{equation*} 
        \item $L$의 치역(range) 혹은 상(image)
            \begin{equation*}
                \Range (L) = \Image (L) = \{L(\vec v) \mid \vec v \in V\}.
            \end{equation*}
        \item $L$이 일대일 함수(one-to-one function) 혹은 단사 함수(injective function, injection)라는 것은 다음을 의미한다:
            \begin{equation*}
                \forall \vec v_1, \vec v_2 \in V \quad (L(\vec v_1) = L(\vec v_2)\ \Rightarrow\ \vec v_1 = \vec v_2)
            \end{equation*}
        \item $L$이 위로의(onto) 함수 혹은 단사 함수(surjective function, surjection)라는 것은 다음을 의미한다:
            \begin{equation*}
                \forall \vec w \in W \quad \exists \vec v \in V \quad L(\vec v) = \vec w
            \end{equation*}
    \end{enumerate}
\end{definition}

\begin{example}
    \leavevmode
    \begin{enumerate}
        \item $L \begin{pmatrix}
                x \\ y \\ z
                \end{pmatrix} = \begin{pmatrix}
                1 & 2 & 3 \\ 4 & 5 & 6 \\ 7 & 8 & 9
                \end{pmatrix} \begin{pmatrix}
                x \\ y \\ z
            \end{pmatrix}$일 때 $\Null (L)$을 구하시오.
            \begin{solution}
                \begin{align*}
                    &\begin{pmatrix}
                    1 & 2 & 3 \\ 4 & 5 & 6 \\ 7 & 8 & 9
                \end{pmatrix}
                &\rightarrow &\begin{pmatrix}
                1 & 2 & 3 \\ 0 & -3 & -6 \\ 0 & -6 & -12
            \end{pmatrix}\\
            \xrightarrow{\text{Echelon}} &\begin{pmatrix}
                1 & 2 & 3 \\ 0 & 1 & 2 \\ 0 & 0 & 0
            \end{pmatrix}
            &\xrightarrow{\text{Reduce}} &\begin{pmatrix}
            1 & 0 & -1 \\ 0 & 1 & 2 \\ 0 & 0 & 0
        \end{pmatrix}
    \end{align*}
    즉
    \begin{equation*}
        \begin{pmatrix}
            1 & 0 & -1 \\ 0 & 1 & 2 \\ 0 & 0 & 0
            \end{pmatrix} \begin{pmatrix}
            x \\ y \\ z
            \end{pmatrix} = \begin{pmatrix}
            0 \\ 0 \\ 0
            \end{pmatrix} \ \Rightarrow \ \begin{pmatrix}
            x \\ y \\ z
            \end{pmatrix} = \begin{pmatrix}
            1 \\ -2 \\ 1
        \end{pmatrix} z
    \end{equation*}
    이 되어, $\Null (L) = \left<(1, -2, 1)\right>$이다.
\end{solution}
    \end{enumerate}
\end{example}
\end{document}
