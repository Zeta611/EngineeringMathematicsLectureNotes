\documentclass[../engineering_mathematics_lecture_note.tex]{subfiles}
\begin{document}
2계 선형 상미분방정식
\begin{equation*}
    L[y] = y'' + p(x) y' + q(x) y = 0
\end{equation*}
의 해공간은 벡터공간을 이루는데, 차원이 2임을 뒤에서 보일 것이다.
따라서, 일차독립인 두 해를 찾으면 그 일차결합이 해공간을 구성하게 된다.

\begin{example}[라그랑주(Lagrange)의 계수 축소법(reduction of order)] \label{def:reduction}
    2계 선형 상미분방정식
    \begin{equation*}
        L[y] = y'' + p(x) y' + q(x) y = 0
    \end{equation*}
    의 해 중 하나가 $y_1(x)$로 알려진 경우, $y_2(x) = u(x)y_1(x)$와 같이 함수 $u(x)$를 두면
    \begin{equation*}
        y_1 u'' + \left( p(x) y_1 + 2y_1' \right) u' = 0
    \end{equation*}
    의 $u'$에 대해 변수 가능한 1계 선형 상미분방정식이 구해진다.
\end{example}

\begin{example}
    \begin{equation*}
        (x^2 - x) y'' - xy' + y = 0
    \end{equation*}
    의 미분방정식의 한 해가 $y_1 = x$임은 쉽게 알 수 있다.
    이제 다른 해를 $ux$라고 두면
    \begin{equation*}
        xu'' + \left( \frac{-x}{x^2 - x} + 2 \right) u' = 0
    \end{equation*}
    이 되고, 정리하면
    \begin{equation*}
        xu'' + \frac{x - 2}{x - 1} u' = 0
    \end{equation*}
    가 되어 변수분리법을 통해 풀 수 있다.
\end{example}

\begin{remark}
    지수함수 $e^{a + bi}$에 대해서 $a, b \in \mathbb R$이고 $i$가 허수일 때,
    \begin{equation*}
        e^{a + bi} = e^a \cdot e^bi = e^a (\cos b + i \sin b)
    \end{equation*}
    이 성립한다.
    따라서, 복소지수 $z_1, z_2$에 대해서도
    \begin{equation*}
        e^{z_1} \cdot e^{z_2} = e^{z_1 + z_2}, \qquad \frac{e_{z_1}}{e_{z_2}} = e^{z_1 - z_2}
    \end{equation*}
    의 지수법칙이 성립한다.

    또, 양의 실수 $c$와 복소수 $z$에 대해서 지수함수 $c^z$를
    \begin{equation*}
        c^z = e^{z \ln c}
    \end{equation*}
    와 같이 정의한다.
\end{remark}

\begin{example}[오일러-코시(Euler-Cauchy) 방정식]
    다음과 같은 선형 동차 상미분방정식을 오일러-코시 방정식이라고 한다:
    \begin{equation*}
        x^2 y'' + ax y' + by = 0
    \end{equation*}
    여기에 $y = x^\alpha$ 꼴을 대입하면
    \begin{equation*}
        \alpha (\alpha - 1) x^2 \cdot x^{\alpha - 2} + a \alpha x \cdot x^{\alpha - 1} + b x^\alpha = 0
    \end{equation*}
    이 되고, 양변에서 $x^\alpha$를 나눠주면
    \begin{equation*}
        \alpha (\alpha - 1) + a\alpha + b = 0
    \end{equation*}
    이 되며 이를 결정방정식(indicial equation)이라고 부른다.

    결정방정식의 해가 두 실근, 중근, 두 허근을 가지는 경우를 각각 살펴본다.
    \begin{enumerate}
        \item 두 실근 $\alpha_1, \alpha_2$를 가지는 경우\\
            두 해
            \begin{equation*}
                y_1 = x^{\alpha_1}, \qquad y_2 = x^{\alpha_2}
            \end{equation*}
            는 독립이므로 일반해 $y$는
            \begin{equation*}
                y = c_1 x^{\alpha_1} + c_2 x^{\alpha_2}
            \end{equation*}
            와 같은 일차결합으로 표현할 수 있다.

        \item 중근 $\alpha$를 가지는 경우, $y_1 = x^\alpha$로 놓고 계수 축소법으로 다른 해를 구한다.

        \item 두 허근 $\alpha_1 = p + qi, \alpha_2 = p - qi$를 가지는 경우\\
            먼저 두 허근으로 표현되는 해 $\tilde y_1, \tilde y_2$를
            \begin{equation*}
                \tilde y_1 = x^{p + qi}, \qquad \tilde y_2 = x^{p - qi}
            \end{equation*}
            으로 놓자.
            변형하면
            \begin{equation*}
                \tilde y_1 = x^p \cdot x^{qi} = x^p e^{q_i \ln x} = x^p \left( \cos (q \ln x) + i\sin(q \ln x) \right)
            \end{equation*}
            와 마찬가지로
            \begin{equation*}
                \tilde y_2 = x^p \left( \cos (q \ln x) - i \sin(q \ln x) \right)
            \end{equation*}
            가 된다.
            두 해를 더하고 빼서 실근으로 만들면
            \begin{equation*}
                y_1 = \frac12 (\tilde y_1 + \tilde y_2), \qquad y_2 = \frac{1}{2i} (\tilde y_1 - \tilde y_2)
            \end{equation*}
            가 되어 일반해
            \begin{equation*}
                y = c_1 y_1 + c_2 y_2
            \end{equation*}
            를 구할 수 있다.
    \end{enumerate}
\end{example}

\begin{example}
    상수계수 제차 선형 2계 상미분방정식
    \begin{equation*}
        y'' + ay' + by = 0
    \end{equation*}
    은 $y = e^{tx}$ 꼴을 해로 가지므로 대입하면,
    \begin{equation*}
        t^2 e^{tx} + at e^{tx} + b e^{tx} = 0
    \end{equation*}
    이 되고, 양변을 $e^{tx}$로 나누면
    \begin{equation*}
        t^2 + at + b = 0
    \end{equation*}
    이 된다.
    이를 특성방정식(characteristic equation)이라고 한다.

    특성방정식의 해가 두 실근, 중근, 두 허근을 가지는 경우를 각각 살펴본다.
    \begin{enumerate}
        \item 두 실근 $t_1, t_2$를 가지는 경우\\
            두 해
            \begin{equation*}
                y_1 = x^{t_1}, \qquad y_2 = x^{t_2}
            \end{equation*}
            는 독립이므로 일반해 $y$는
            \begin{equation*}
                y = c_1 e^{t_1 x} + c_2 e^{t_2 x}
            \end{equation*}
            와 같은 일차결합으로 표현할 수 있다.

        \item 중근 $t$를 가지는 경우\\
            한 해 $y_1 = e^{tx}$로 놓고 계수 축소법으로 다른 해를 구하면 $y_2 = x e^{tx}$가 된다.
            따라서 일반해는
            \begin{equation*}
                y = c_1 e^{tx} + c_2 x e^{tx}
            \end{equation*}
            와 같이 표현된다.

        \item 두 허근 $t_1 = \alpha + \beta i, t_2 = \alpha - \beta i$를 가지는 경우\\
            먼저 두 허근으로 표현되는 해 $\tilde y_1, \tilde y_2$를
            \begin{equation*}
                \tilde y_1 = e^{\alpha + \beta i}, \qquad \tilde y_2 = e^{\alpha - \beta i}
            \end{equation*}
            으로 놓자.
            오일러-코시 방정식의 경우와 마찬가지로 두 해를 더하고 빼서 실근으로 만들면
            \begin{equation*}
                y_1 = \frac12 (\tilde y_1 + \tilde y_2), \qquad y_2 = \frac{1}{2i} (\tilde y_1 - \tilde y_2)
            \end{equation*}
            가 되어 일반해
            \begin{equation*}
                y = c_1 y_1 + c_2 y_2
            \end{equation*}
            를 구할 수 있다.
    \end{enumerate}
\end{example}

\begin{example}
    먼저\marginpar{2018.10.29} 다음의 상수계수 제차 선형 2계 상미분방정식
    \begin{equation*}
        y'' + 4y' + 4y = 0
    \end{equation*}
    은 특성방정식
    \begin{equation*}
        t^2 + 4t + 4 = 0
    \end{equation*}
    를 가진다.
    이는 중근 $t = -2$를 가지므로 두 해
    \begin{equation*}
        y_1 = e^{-2x},\qquad y_2 = xe^{-2x}
    \end{equation*}
    를 해로 가진다.
    만약 $y_2$를 계수 축소법을 통해 푼다면, $y_2 = u(x) y_1$으로 놓을 수 있다.
    원 미분방정식에 대입하면
    \begin{equation*}
        \left( u'' y_1 + 2 u' y_1' + u y_1'' \right) + \left( 4u' y_1 + 4uy_1' \right) + 4uy_1 = 0
    \end{equation*}
    가 되고,
    \begin{equation*}
        uy_1'' + 4u y_1' + 4u y_1 = 0
    \end{equation*}
    이므로
    \begin{equation*}
        u'' y_1 + 2u' y_1' + 4u' y_1 = 0
    \end{equation*}
    으로 정리된다.
    $y_1 = e^{-2x}$이므로 대입하면
    \begin{equation*}
        u'' e^{-2x} - 4 u' e^{-2x} + 4u' e^{-2x} = u'' e^{-2x} = 0
    \end{equation*}
    이므로 $u'' = 0$이다.
    이를 만족하는 $u$ 중에는 $u = x$가 있으므로,
    \begin{equation*}
        y_2 = x e^{-2x}
    \end{equation*}
    로 결정할 수 있다.

    따라서 일반해는
    \begin{equation*}
        y = c_1 e^{-2x} + c_2 xe^{-2x}
    \end{equation*}
    이다.
\end{example}

\begin{example}
    위 예시의 제차 상미분방정식에서 초항을 $x$로 두자:
    \begin{equation*}
        y'' + 4y' + 4y = x.
    \end{equation*}
    위에서 일반해 $y_h$를
    \begin{equation*}
        y_h = c_1 e^{-2x} + c_2 xe^{-2x}
    \end{equation*}
    로 찾았으므로 특수해 $y_p$를 찾으면 된다.

    먼저
    \begin{equation*}
        y_p = ax + b
    \end{equation*}
    꼴을 시도하면,
    \begin{equation*}
        y_p'' = 0,\qquad y_p' = a
    \end{equation*}
    이므로 원 미분방정식에 대입하면
    \begin{equation*}
        4a + 4ax + 4b = x
    \end{equation*}
    의 항등식이 나와
    \begin{equation*}
        a = -b = \frac14
    \end{equation*}
    가 된다.
    따라서 특수해 $y_p$는
    \begin{equation*}
        y_p = \frac14 x - \frac14
    \end{equation*}
    로 결정할 수 있고, 일반해 $y$는
    \begin{equation*}
        y = c_1 e^{-2x} + c_2 xe^{-2x} + \frac14 x - \frac14
    \end{equation*}
    이다.
\end{example}

\begin{example}
    위 예시의 초항을 $x$에서 $e^x$로 바뀌면,
    \begin{equation*}
        y'' + 4y' + 4y = e^x
    \end{equation*}
    가 되고, 특수해
    \begin{equation*}
        y_p = Ae^x
    \end{equation*}
    로 두고 풀 수 있다.

    반면 초항을 $e^{-2x}$로 두면 특수해를
    \begin{equation*}
        y_p = Ae^{-2x}
    \end{equation*}
    혹은
    \begin{equation*}
        y_p = Axe^{-2x}
    \end{equation*}
    로 시도하면, 원 미분방정식의 좌변이 0이 된다.
    그러나
    \begin{equation*}
        y_p = Ax^2 e^{-2x}
    \end{equation*}
    를 대입하면 풀 수 있다.
\end{example}

\begin{example}
    일반적으로 상수계수 비제차 선형 상미분방정식
    \begin{equation*}
        y'' + ay' + by = r(x)
    \end{equation*}
    에 대해서, $r(x)$가
    \begin{enumerate}
        \item 다항식인 경우 $y_p(x)$를 다항식으로,
        \item $\sin x$ 혹은 $\cos x$인 경우 $y_p(x) = A \cos x + B \sin x$로,
        \item $e^{px} \cos (qx)$인 경우 $y_p(x) = A e^{px} \cos (qx) + B e^{px} \sin (qx)$로
    \end{enumerate}
    두고 시도할 수 있다.
\end{example}

\begin{theorem}
    2계 상미분방정식
    \begin{equation*}
        y'' + p(x) y' + q(x) y = r(x)
    \end{equation*}
    에 대해서
    $p, q, r$이 $I \subset \mathbb R$에서 연속인 실함수일 때, 해 $y: I \rightarrow \mathbb R$이 존재한다.
\end{theorem}

\begin{theorem} [초기값 문제(Initial Value Problem)] \label{thm:ivp}
    2계 상미분방정식
    \begin{equation*}
        y'' + p(x) y' + q(x) y = r(x)
    \end{equation*}
    에 대해서
    $p, q, r$이 $I \subset \mathbb R$에서 연속인 실함수일 때,
    \begin{equation*}
        y(a) = y_0, \qquad y'(a) = y_1
    \end{equation*}
    을 만족하는 $y: I \rightarrow \mathbb R$이 유일하게 존재한다.
\end{theorem}

\begin{definition} [론스키안(Wronskian)]
    미분방정식의 해 $y_1(x), \dots, y_k(x)$에 대해서 론스키안 $W$는
    \begin{equation*}
        W(y_1, \dots, y_k) = \det \begin{pmatrix}
            y_1 & \dots & y_k\\
            y_1' & \dots & y_k'\\
            \vdots & \ddots & \vdots\\
            y_k^{(k - 1)} & \dots & y_k^{(k - 1)}
        \end{pmatrix}_{k \times k}
    \end{equation*}
    로 정의한다.
\end{definition}

\begin{theorem} \label{thm:wronskian_independence}
    제차 2계 상미분방정식
    \begin{equation*}
        L[y] = y'' + p(x) y' + q(x) y = 0
    \end{equation*}
    에 대해서
    $p, q, r$이 $I \subset \mathbb R$에서 연속인 실함수일 때, $y_1, y_2$를 해로 가진다고 하자.
    이때 다음의 명제들이 동치다:
    \begin{enumerate}
        \item 어떤 $x_0 \in I$에 대해 $W(y_1, y_2)(x_0) = 0$이다.
        \item 모든 $x \in I$에 대해 $W(y_1, y_2)(x) = 0$이다.
        \item $y_1$과 $y_2$가 일차종속이다.
    \end{enumerate}
\end{theorem}

\begin{proof}
    우선 2에서 1임은 당연하다.
    1에서 3임을 보이자.

    론스키안이 0이므로,
    \begin{equation*}
        \begin{pmatrix}
            y_1(x_0) & y_2(x_0)\\
            y_1'(x_0) & y_2'(x_0)
        \end{pmatrix} \begin{pmatrix}
            c_1 \\ c_2
        \end{pmatrix} = \begin{pmatrix}
            0 \\ 0
        \end{pmatrix}
    \end{equation*}
    을 만족하는 비자명한 $(c_1, c_2)$가 존재한다.
    이러한 $(c_1, c_2)$에 대해
    \begin{equation*}
        y = c_1 y_1 + c_2 y_2
    \end{equation*}
    를 놓으면, $y(x_0) = y'(x_0) = 0$을 만족하므로 정리~\ref{thm:ivp}에 의해 초기값 문제
    \begin{equation*}
        L[y] = 0, \qquad y(x_0) = y'(x_0) = 0
    \end{equation*}
    의 유일한 해이다.
    그런데 영함수도 위 초기값 문제를 만족하므로 $y$는 영함수이다.
    따라서
    \begin{equation*}
        c_1 y_1(x_0) + c_2 y_2(x_0) = 0
    \end{equation*}
    에서 $c_1$과 $c_2$가 동시에 0이 아니므로 $y_1$과 $y_2$는 일차종속이다.

    이제 3에서 2를 보이자.

    $y_1$과 $y_2$가 종속이므로
    \begin{equation*}
        c_1 y_1 + c_2 y_2 = 0
    \end{equation*}
    을 만족하는 비자명한 $(c_1, c_2)$가 존재한다.
    이러한 $(c_1, c_2)$에 대해
    \begin{equation*}
        c_1 y_1' + c_2 y_2' = 0
    \end{equation*}
    을 만족하므로
    \begin{equation*}
        \begin{pmatrix}
            y_1 & y_2\\
            y_1' & y_2'
        \end{pmatrix} \begin{pmatrix}
            c_1 \\ c_2
        \end{pmatrix} = \begin{pmatrix}
            0 \\ 0
        \end{pmatrix}
    \end{equation*}
    에서 론스키안이 0이다.
\end{proof}

\begin{remark}
    오일러-코시 방정식
    \begin{equation*}
        x^2 y'' - 2xy' + 2y = 0
    \end{equation*}
    에 대해서 두 해
    \begin{equation*}
        y_1 = x, \qquad y_2 =x^2
    \end{equation*}
    은 일차독립이다.
    이때 론스키안
    \begin{equation*}
        \det
        \begin{pmatrix}
            x & x^2\\
            1 & 2x
        \end{pmatrix}
        = x^2
    \end{equation*}
    은 $x = 0$에서 0이므로 정리~\ref{thm:wronskian_independence}에 상충되는 것으로 보인다.
    그러나 정리~\ref{thm:wronskian_independence}에서 다룬 형태와 같이 $y''$의 계수를 1로 변형하면
    \begin{equation*}
        y'' - \frac2x y' + \frac{2}{x^2} y = 0
    \end{equation*}
    이 되고, 이는 $x = 0$에서 불연속이므로 정리와 모순되지 않는다.
\end{remark}

\begin{theorem} \label{thm:2nd_order_sol_space_at_least_dim_2}
    제차 2계 상미분방정식
    \begin{equation*}
        L[y] = y'' + p(x) y' + q(x) y = 0
    \end{equation*}
    에 대해서
    $p, q, r$이 $I \subset \mathbb R$에서 연속인 실함수일 때, 해공간의 차원은 2보다 작지 않다.
\end{theorem}

\begin{proof}
    두 초기값 문제
    \begin{equation*}
        L[y] = 0, \quad y(x_0) = 1, \quad y'(x_0) = 0
    \end{equation*}
    과
    \begin{equation*}
        L[y] = 0, \quad y(x_0) = 0, \quad y'(x_0) = 1
    \end{equation*}
    의 해는 각각 어떤 $y_1$와 $y_2$로 유일하게 존재한다.
    그런데 론스키안 $W(y_1, y_2)(x_0)$는
    \begin{equation*}
        W(y_1, y_2)(x_0) = \det \begin{pmatrix}
            1 & 0\\
            0 & 1
        \end{pmatrix}
        = 1 \neq 0
    \end{equation*}
    이므로 $y_1$과 $y_2$는 일차독립이다.
    따라서 $L[y] = r(x)$는 적어도 두 개의 일차독립인 해를 가지므로 차원이 적어도 2 이상이다.
\end{proof}

\begin{theorem}
    제차 2계 상미분방정식
    \begin{equation*}
        L[y] = y'' + p(x) y' + q(x) y = 0
    \end{equation*}
    에 대해서
    $p, q, r$이 $I \subset \mathbb R$에서 연속인 실함수일 때, 해공간의 차원은 2이다.
\end{theorem}

\begin{proof}
    $L[y] = 0$는 정리~\ref{thm:2nd_order_sol_space_at_least_dim_2}에 따라 일차독립인 두 해 $y_1$과 $y_2$를 가진다.
    이제 $y_1$과 $y_2$와 다른 어떤 해 $y$를 잡자.
    이렇게 잡은 $y$가 $y_1$과 $y_2$의 일차결합으로 나태내어진다는 것을 보이자.

    방정식
    \begin{equation*}
        \begin{pmatrix}
            y_1(x_0) & y_2(x_0)\\
            y_1'(x_0) & y_2'(x_0)
        \end{pmatrix} \begin{pmatrix}
            x \\ y
        \end{pmatrix}
        = \begin{pmatrix}
            y(x_0) \\ y'(x_0)
        \end{pmatrix}
    \end{equation*}
    에서 계수 행렬의 행렬식은 $y_1$과 $y_2$에 대한 론스키안이므로 정리~\ref{thm:wronskian_independence}에 의해 0이 아니다.
    따라서 비자명한 해 $(x, y) = (c_1, c_2)$가 존재한다.
    즉,
    \begin{align*}
        y(x_0) &= c_1 y_1(x_0) + c_2 y_2(x_0)\\
        y'(x_0) &= c_1 y'_1(x_0) + c_2 y'_2(x_0)
    \end{align*}
    이 성립하여 $y$와 $c_1 y_1 + c_2 y_2$가 동일한 초기값 문제의 해이다.
    그러므로 정리~\ref{thm:ivp}에 따라
    \begin{equation*}
        y = c_1 y_1 + c_2 y_2
    \end{equation*}
    이다.
    결국 $L[y] = 0$의 모든 해는 $y_1$과 $y_2$의 일차결합으로 표현할 수 있는 2차원 벡터공간이다.
\end{proof}

\begin{example}[라그랑주의 매개변수 변환법(Variation of Parameters)]
    비제차\marginpar{2018.10.31} 상미분방정식
    \begin{equation*}
        L[y] = y'' + p(x) y' + q(x) y = r(x)
    \end{equation*}
    가 주어졌을 때, 우선 $y_1$과 $y_2$가 $L[y] = 0$의 제차 방정식 일차독립인 두 해라고 하자.
    즉 $L[y] = 0$의 일반해 $y$는 $c_1 y_1 + c_2 y_2$로 나타낼 수 있다.
    
    매개변수 변환법은 비제차 상미분방정식 $L[y] = r(x)$의 특수해 $y_p$를
    \begin{equation*}
        y_p = u(x) y_1 + v(x) y_2
    \end{equation*}
    와 같이 나타내는 방법을 말한다.
    양변을 미분하면
    \begin{equation*}
        y_p' = \left( u' y_1 + u y_1' \right) + \left( v' y_2 + v y_2' \right)
    \end{equation*}
    가 되는데, 특히
    \begin{equation*}
        u' y_1 + v' y_2 = 0
    \end{equation*}
    인 경우를 다룬다.
    따라서
    \begin{equation*}
        y_p'= u y_1' + v y_2'
    \end{equation*}
    로 정리할 수 있고, 양변을 한 번 더 미분하면
    \begin{equation*}
        y_p'' = u' y_1' + u y_1'' + v' y_2' + v y_2''
    \end{equation*}
    가 된다.
    원 미분방정식에 대입하면
    \begin{align*}
        L[y_p] &= \left( u' y_1' + u y_1'' + v' y_2' + v y_2'' \right) + p(x) \left( uy_1' + vy_2' \right) + q(x) (u y_1 + v y_2)\\
               &= \left( u y_1'' + u y_1' + u y_1 \right) + \left( v y_2'' + v y_2' + v y_2 \right) + \left( u' y_1' + v' y_2' \right)\\
               &= u' y_1' + v' y_2' = r(x)
    \end{align*}
    이 된다.
    따라서
    \begin{align*}
        u' y_1 + v' y_2 &= 0\\
        u' y_1' + v' y_2' &= r
    \end{align*}
    이 되고, 행렬로 표현하면
    \begin{equation*}
        \begin{pmatrix}
            y_1 & y_2\\
            y_1' & y_2'
        \end{pmatrix}
        \begin{pmatrix}
            u' \\ v'
        \end{pmatrix}
        = \begin{pmatrix}
            0 \\ r
        \end{pmatrix}
    \end{equation*}
    이다.
    $y_1$과 $y_2$는 일차독립이므로 정의역의 모든 $x$에 대해 론스키언, 즉 행렬
    \begin{equation*}
        \begin{pmatrix}
            y_1 & y_2\\
            y_1' & y_2'
        \end{pmatrix}
    \end{equation*}
    의 행렬식이 0이 아니다.
    따라서 위 행렬은 가역행렬이고,
    \begin{equation*}
        \begin{pmatrix}
            u' \\ v'
        \end{pmatrix}
        =
        \begin{pmatrix}
            y_1 & y_2\\
            y_1' & y_2'
        \end{pmatrix}^{-1}
        \begin{pmatrix}
            0 \\ r
        \end{pmatrix}
        = \frac{1}{W(y_1, y_2)} \begin{pmatrix}
            -ry_2 \\ ry_1
        \end{pmatrix}
    \end{equation*}
    이다.
\end{example}

\begin{example}
    비제차 상미분방정식
    \begin{equation*}
        y'' + y = \sec x
    \end{equation*}
    의 일반해를 구하기 위해, 먼저 특성방정식
    \begin{equation*}
        t^2 + 1 = 0
    \end{equation*}
    의 해 $t = \pm i$를 통해 제차 상미분방정식의 해 $y_h$를
    \begin{equation*}
        y_h = c_1 \cos x + c_2 \sin x
    \end{equation*}
    를 구할 수 있다.

    위의 매개변수 변환법을 통해 특수해 $y_p$를
    \begin{equation*}
        y_p = u_1 \cos x + u_2 \sin x
    \end{equation*}
    로 놓자.
    그러면 론스키언은
    \begin{equation*}
        W(y_1, y_2) = \det \begin{pmatrix}
            \cos x & \sin x\\
            - \sin x & \cos x
        \end{pmatrix} = 1
    \end{equation*}
    이 된다.
    따라서
    \begin{equation*}
        u' = \frac{1}{W(y_1, y_2)} (-r y_2) = - \sec x \sin x = -\tan x = \frac{- \sin x}{\cos x} 
    \end{equation*}
    이고,
    \begin{equation*}
        u = \ln \left| \cos x \right|
    \end{equation*}
    를 알 수 있다.
    또,
    \begin{equation*}
        v' = \frac{1}{W(y_1, y_2)} (r y_1) = \sec x \cos x = 1
    \end{equation*}
    이므로 $v = x$를 결정할 수 있다.
    그러므로 특수해 $y_p$는
    \begin{equation*}
        y_p = \cos x \ln \left| \cos x \right| + x \sin x
    \end{equation*}
    이며, 따라서 일반해 $y$는
    \begin{equation*}
        y = (c_1 + \ln \left| \cos x \right|) \cos x + (c_2 + x) \sin x
    \end{equation*}
    로 나타낼 수 있다.
\end{example}
\end{document}
